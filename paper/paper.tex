\documentclass[preprint, 3p,
authoryear]{elsarticle} %review=doublespace preprint=single 5p=2 column
%%% Begin My package additions %%%%%%%%%%%%%%%%%%%

\usepackage[hyphens]{url}

  \journal{Sustainable Cities and Society} % Sets Journal name

\usepackage{graphicx}
%%%%%%%%%%%%%%%% end my additions to header

\usepackage[T1]{fontenc}
\usepackage{lmodern}
\usepackage{amssymb,amsmath}
% TODO: Currently lineno needs to be loaded after amsmath because of conflict
% https://github.com/latex-lineno/lineno/issues/5
\usepackage{lineno} % add
\usepackage{ifxetex,ifluatex}
\usepackage{fixltx2e} % provides \textsubscript
% use upquote if available, for straight quotes in verbatim environments
\IfFileExists{upquote.sty}{\usepackage{upquote}}{}
\ifnum 0\ifxetex 1\fi\ifluatex 1\fi=0 % if pdftex
  \usepackage[utf8]{inputenc}
\else % if luatex or xelatex
  \usepackage{fontspec}
  \ifxetex
    \usepackage{xltxtra,xunicode}
  \fi
  \defaultfontfeatures{Mapping=tex-text,Scale=MatchLowercase}
  \newcommand{\euro}{€}
\fi
% use microtype if available
\IfFileExists{microtype.sty}{\usepackage{microtype}}{}
\usepackage[]{natbib}
\bibliographystyle{plainnat}

\ifxetex
  \usepackage[setpagesize=false, % page size defined by xetex
              unicode=false, % unicode breaks when used with xetex
              xetex]{hyperref}
\else
  \usepackage[unicode=true]{hyperref}
\fi
\hypersetup{breaklinks=true,
            bookmarks=true,
            pdfauthor={},
            pdftitle={Which fifteen-minutes neighborhoods are dead-ends? An analysis of the network attributes of fifteen-minute pedsheds},
            colorlinks=false,
            urlcolor=blue,
            linkcolor=magenta,
            pdfborder={0 0 0}}

\setcounter{secnumdepth}{5}
% Pandoc toggle for numbering sections (defaults to be off)


% tightlist command for lists without linebreak
\providecommand{\tightlist}{%
  \setlength{\itemsep}{0pt}\setlength{\parskip}{0pt}}







\begin{document}


\begin{frontmatter}

  \title{Which fifteen-minutes neighborhoods are dead-ends? An analysis
of the network attributes of fifteen-minute pedsheds}
    \author[University One]{Author One%
  \corref{cor1}%
  \fnref{1}}
   \ead{a1@example.com} 
    \author[University Two]{Author Two%
  %
  }
   \ead{a2@example.com} 
    \author[University One]{Author Three%
  %
  \fnref{2}}
   \ead{a3@example.com} 
    \author[University One]{Author Four%
  %
  \fnref{2}}
   \ead{a4@example.com} 
      \affiliation[University One]{
    organization={Department},addressline={1 main
street},city={City},postcode={123456},state={State},country={Country},}
    \affiliation[University Two]{
    organization={Department},addressline={2 main
street},city={City},postcode={2054},country={Country},}
    \cortext[cor1]{Corresponding author}
    \fntext[1]{This is the first author footnote.}
    \fntext[2]{Another author footnote.}
  
  \begin{abstract}
  Fifteen-minutes neihborhoods, a form of normative chronourbanism based
  on cumulative opportunities, has gained recognized as a way to reduce
  the need for motorized travel, and increase the livability,
  convenience, and health of the public. At the core of this concept is
  a pedshed, an area defined by the walkable isochrone of the eponymous
  fifteen minutes. As the idea of fifteen-minutes neighborhoods develops
  traction in policy and planning circles, it seems timely to revisit
  the way street network design can support---or obstruct---the stated
  goal of preserving or creating walkable neighborhoods with essential
  amenities. In this paper we examine a sample (\(n=834\)) of
  fifteen-minutes pedsheds in Hamilton, a medium-sized city in Canada,
  and how their sizes relate to the attributes of the transportation
  network. The analysis reveals that network design in suburban Hamilton
  conspires against the creating of fifteen-minutes neighborhoods. Much
  of urban Hamilton, in contrast, already has the characteristics of
  fifteen-minutes neighborhoods. The research points to elements of
  network design that can help to discriminate between candidate
  neighborhoods and dead-ends, and that can provide parameters for the
  design of new developments.
  \end{abstract}
    \begin{keyword}
    Fifteen-minutes
neighborhoods \sep Pedshed \sep Walkability \sep Accessibility \sep 
    Network analysis
  \end{keyword}
  
 \end{frontmatter}

\hypertarget{introduction}{%
\section{Introduction}\label{introduction}}

``Il faut oublier la traversée de Paris d'est en ouest en
voiture''\footnote{ ``We must forget about crossing Paris from east to
  west by car''} \citep{alimi2020}.

\citet{knight2018walkable} \citet{liu2022toward}
\citet{pozoukidou2021fifteen} \citet{weng2019fifteen}

\renewcommand\refname{References}
\bibliography{../bibliography/bibliography.bib}


\end{document}
