% !TeX program = pdfLaTeX
\documentclass[smallextended]{svjour3}       % onecolumn (second format)
%\documentclass[twocolumn]{svjour3}          % twocolumn
%
\smartqed  % flush right qed marks, e.g. at end of proof
%
\usepackage{amsmath}
\usepackage{graphicx}
\usepackage[utf8]{inputenc}

\usepackage[hyphens]{url} % not crucial - just used below for the URL
\usepackage{hyperref}

%
% \usepackage{mathptmx}      % use Times fonts if available on your TeX system
%
% insert here the call for the packages your document requires
%\usepackage{latexsym}
% etc.
%
% please place your own definitions here and don't use \def but
% \newcommand{}{}
%
% Insert the name of "your journal" with
% \journalname{myjournal}
%

%% load any required packages here



% tightlist command for lists without linebreak
\providecommand{\tightlist}{%
  \setlength{\itemsep}{0pt}\setlength{\parskip}{0pt}}


% Pandoc citation processing
\newlength{\cslhangindent}
\setlength{\cslhangindent}{1.5em}
\newlength{\csllabelwidth}
\setlength{\csllabelwidth}{3em}
\newlength{\cslentryspacingunit} % times entry-spacing
\setlength{\cslentryspacingunit}{\parskip}
% for Pandoc 2.8 to 2.10.1
\newenvironment{cslreferences}%
  {}%
  {\par}
% For Pandoc 2.11+
\newenvironment{CSLReferences}[2] % #1 hanging-ident, #2 entry spacing
 {% don't indent paragraphs
  \setlength{\parindent}{0pt}
  % turn on hanging indent if param 1 is 1
  \ifodd #1
  \let\oldpar\par
  \def\par{\hangindent=\cslhangindent\oldpar}
  \fi
  % set entry spacing
  \setlength{\parskip}{#2\cslentryspacingunit}
 }%
 {}
\usepackage{calc}
\newcommand{\CSLBlock}[1]{#1\hfill\break}
\newcommand{\CSLLeftMargin}[1]{\parbox[t]{\csllabelwidth}{#1}}
\newcommand{\CSLRightInline}[1]{\parbox[t]{\linewidth - \csllabelwidth}{#1}\break}
\newcommand{\CSLIndent}[1]{\hspace{\cslhangindent}#1}

\begin{document}


\title{Reality or wishful thinking: Examining the feasibility of
15-minute neighborhoods from the perspective of network
attributes \thanks{This project was supported by a McMaster Institute
for Research on Aging Catalyst Grant.} }


    \titlerunning{Examinining fifteen minute neighborhoods from the
perspective of network attributes}

\author{  Antonio Paez \and  James R. Dunn \and  Josh Arbess \and  }


\institute{
        Antonio Paez \at
     School of Earth, Environment and Society, McMaster University \\
     \email{\href{mailto:paezha@mcmaster.ca}{\nolinkurl{paezha@mcmaster.ca}}}  %  \\
%             \emph{Present address:} of F. Author  %  if needed
    \and
        James R. Dunn \at
     Department of Health, Aging and Society, McMaster University \\
     \email{\href{mailto:jim.dunn@mcmaster.ca}{\nolinkurl{jim.dunn@mcmaster.ca}}}  %  \\
%             \emph{Present address:} of F. Author  %  if needed
    \and
        Josh Arbess \at
     Mechanical Engineering and Society, McMaster University \\
     \email{\href{mailto:arbessj@mcmaster.ca}{\nolinkurl{arbessj@mcmaster.ca}}}  %  \\
%             \emph{Present address:} of F. Author  %  if needed
    \and
    }

\date{Received: date / Accepted: date}
% The correct dates will be entered by the editor


\maketitle

\begin{abstract}
Walking has been the primary form of personal transportation for as long
as humans have been humans. From prehistoric migration to the dawn of
populated settlements, the space-time constraints of walking have
fundamentally determined how far interactions with the environment and
other humans were feasible. It took tens of thousands of years for
humans to walk out from their original African niche to eventually
settle every nook of the planet.

The advent of permanent settlements imposed stronger space-time
constraints on people's movement than had hitherto been the case. The
need to return to a fixed abode or to go to a designated place to drudge
or trade meant that people become more local of necessity. Despite the
invention of the wheel and the use of animal power for transportation
purposes, humanity remained for the most part pedestrian for millennia
(Roberts 1998). The physical effort involved in walking limited the
geographical extent of settlements and in turn settlements evolved to
facilitate walking.

The dominance of travel by foot faded within the span of a few decades.
Several revolutions contributed to this. Technological innovations in
the 19th and 20th centuries led to the internal combustion engine,
smooth paving surfaces, and systems to transport. Concurrently,
socio-technological innovations (e.g., Fordism) created the basis for
mass production and consumption. The pioneer in this respect was of
course the auto industry; many decades later, whole economic systems are
still dominated by this sector (Jane Jacobs famously quipped that
``{[}t{]}he purpose of life is to produce and consume automobiles.'')
Early in the automotive era cars were seen as an ideal solution to many
urban ills (Brown, Morris, and Taylor 2009), which contributes to
explain their enthusiastic reception. In the second part of the 20th
century, motorized mobility rose to become the dominant form of
transportation in cities and regions around the world. The automobile
replaced walking as the key determinant of how far settlements could
grow, and cities grew to accommodate this form of transportation, often
with unfortunate single-mindedness.

The results of this process have been calamitous.

Reliance on motorized mobility contributes to pollution, climate change,
the erosion of social capital and sense of community, road fatalities,
and poor health due to sedentary lifestyles. For years now, work has
aimed to grow a consensus about the importance of communities that
better serve all their residents, and not only their vehicles. The
15-minute neighborhood {[}15MN; Pozoukidou and Chatziyiannaki (2021){]}
is among a handful of ideas that strive to emphasize movement at a human
scale, in environments that accommodate a wide range of capabilities
throughout the lifespan, with the aim of improving livability and health
in ways that automobility can no longer promise, let alone deliver.

A challenge faced by 15MNs is the legacy of decades of auto-centric
planning. Streetscapes are key parts of the hardware of cities, not only
for what is evident at surface level (e.g., sidewalks, pavements), but
also due to other accessory but hidden infrastructure, both physical
(e.g., water, sewage, power) and social (e.g., property rights, right of
ways). Suburban developments in North America are often implicitly or
explicitly designed to discourage through-traffic. This is done by
creating predominantly single use landscapes with meandering, poorly
connected roads. Alas, this form of development cuts both ways, since
the high built-in cost of navigation does not discriminate between
outsiders and residents.

A relevant question concerns the kinds of streetscapes that can support
15MNs. To address this question, we investigate the current
accessibility situation in parts of Canada's major metropolitan region.
The analysis consists of two parts, with positive and normative
characters. 15-minute walking neighborhoods are studied, and their
accessibility levels assessed (positive analysis). Optimal opportunity
landscapes are then used to simulate equivalent opportunity landscapes
throughout the region. Accessibility is then reanalyzed from the
normative perspective of the provision of opportunities. The results of
this analysis are finally correlated to neighborhood network attributes,
including connectivity, centrality, and clustering. The results of this
investigation provide valuable information about neighborhoods, their
morphology and potential to support the aspirational goal of providing
opportunities within 15-minute walks for their residents. This
information can help to identify target neighborhoods for planning
interventions, as well as neighborhoods for whom the 15-minute ideal
could be little more than wishful thinking.
\\
\keywords{
        Walking \and
        Accessibility \and
        15-minute neighborhood \and
    }


\end{abstract}


\def\spacingset#1{\renewcommand{\baselinestretch}%
{#1}\small\normalsize} \spacingset{1}


\hypertarget{references}{%
\subsection*{References}\label{references}}
\addcontentsline{toc}{subsection}{References}

\hypertarget{refs}{}
\begin{CSLReferences}{1}{0}
\leavevmode\vadjust pre{\hypertarget{ref-brown2009planning}{}}%
Brown, J. R., E. A. Morris, and B. D. Taylor. 2009. {``Planning for Cars
in Cities: Planners, Engineers, and Freeways in the 20th Century.''}
Journal Article. \emph{Journal of the American Planning Association} 75
(2): 161--77. \url{https://doi.org/10.1080/01944360802640016}.

\leavevmode\vadjust pre{\hypertarget{ref-pozoukidou2021fifteen}{}}%
Pozoukidou, Georgia, and Zoi Chatziyiannaki. 2021. {``15-Minute City:
Decomposing the New Urban Planning Eutopia.''} Journal Article.
\emph{Sustainability} 13 (2): 928.
\url{https://doi.org/10.3390/su13020928}.

\leavevmode\vadjust pre{\hypertarget{ref-roberts1998short}{}}%
Roberts, Ian. 1998. {``A Short History of Walking.''} Journal Article.
\emph{Nature Medicine} 4 (3): 263--64.
\url{https://doi.org/10.1038/nm0398-263}.

\end{CSLReferences}


\bibliographystyle{spphys}
\bibliography{bibliography.bib}


\end{document}
